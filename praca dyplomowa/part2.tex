\chapter{Obliczenia projektowe}

\section{Obliczenia i wzory}

\subsection{Przemiana politropowa}
Dla przemiany politropowej prawdziwa jest zależność:

\begin{equation}
p \cdot V^n = idem
\label{eq:number_4}
\end{equation}

Aby wyeliminować objętość należy spierwiastkować równanie \eqref{eq:number_4}, co prowadzi do postaci:

\[p^{\frac {1}{n}} \cdot V = idem\]

Równanie Clapeyrona przekształcone celem uzyskania objętości po lewej stronie ma postać:

\[V = \frac{nRT} {p}\]

Zestawiając obie równości uzyskuje się:

\[p^{\frac {1}{n}} \cdot \frac{nRT} {p} = idem\]

Po podzieleniu obu stron równania przez $nR$ (stała) oraz zapisaniu $1/p$ jako $p^{-1}$ równanie przyjmuje postać:

\[p^{\frac {1}{n}} \cdot p^{-1} \cdot T = idem\]

Ponieważ $x^a \cdot x^b = x^{a+b}$ uporządkowanie powyższego równania prowadzi ostatecznie do zależności pomiędzy temperaturą i ciśnieniem przemiany politropowej:

\[p^{\frac {1-n}{n}} \cdot T = idem\]

Zakładając, że przemiana politropowa przebiega pomiędzy stanami 1 i 2, można zapisać następującą równość:

\[p_1 ^{\frac {1-n}{n}} \cdot T_1 = p_2 ^{\frac {1-n}{n}} \cdot T_2\]

Co daje się uporządkować do następujących postaci:

\begin{equation}
\frac{T_1}{T_2} =   \left( \frac {p_2}{p_1}\right) ^{\frac {1-n}{n}}
\label{eq:number_5}
\end{equation}

\begin{equation}
\frac{p_2}{p_1} =   \left( \frac {T_2}{T_1}\right) ^{\frac {n}{n-1}}
\label{eq:number_6}
\end{equation}


\subsection{Obliczenie wykładnika przemiany }
Aby możliwe było obliczenie wykładnika przemiany, konieczne jest przekształcenie równania \eqref{eq:number_5}. Zlogarytmowanie obu stron równania prowadzi do następującej postaci:

\[log \left( \frac{T_1}{T_2} \right) = log \left( \frac{p_2}{p_1} \right)^ {\left( \frac{1-n}{n} \right)} \]

Ponieważ $log(a)^b = b\cdot log(a)$:

\[log \left( \frac{T_1}{T_2} \right) = \left( \frac{1-n}{n} \right) \cdot log \left( \frac{p_2}{p_1} \right)\]

Po pomnożeniu obu stron przez $n$:

\[n \cdot log \left( \frac{T_1}{T_2} \right) = \left( {1-n} \right) \cdot log \left( \frac{p_2}{p_1} \right)\]

Wykonujemy mnożenie po prawej stronie:

\[n \cdot log \left( \frac{T_1}{T_2} \right) = log \left( \frac{p_2}{p_1} \right) - n \cdot log \left( \frac{p_2}{p_1} \right)\]

Przenosimy iloczyn przez $n$ na lewą stronę i wyciągamy $n$ przed nawias:

\[n \cdot \left[ log \left( \frac{T_1}{T_2} \right) + log \left( \frac{p_2}{p_1} \right) \right] = log \left( \frac{p_2}{p_1} \right)\]

Sumę logarytmów można zapisać jako logarytm iloczynu:

\[n \cdot \left[ log \left( \frac{T_1}{T_2} \cdot \frac{p_2}{p_1} \right) \right] = log \left( \frac{p_2}{p_1} \right)\]

Ostatecznie wykładnik politropy oblicza się z następującej zależności:

\[n = \frac {log \left( \frac{p_2}{p_1} \right)} {log \left( \frac{T_1}{T_2} \cdot \frac{p_2}{p_1} \right)}  \]



\subsection{Praca przemiany politropowej}
W termodynamice pracę $L_{1-2}$ oblicza się całkując:

\begin{equation}
L_{1-2} = \int_1^2 p\cdot dV
\label{eq:number_7}
\end{equation}


Wiedząc, że dla przemiany politropowej obowiązuje równanie \eqref{eq:number_4} można zapisać:

\begin{equation}
p = \frac{idem}{V^n}
\label{eq:number_8}
\end{equation}
Co po podstawieniu do równania \eqref{eq:number_7} daje:

\[L_{1-2} = \int_1^2 {\frac{idem}{V^n} \cdot dV} \]

Po uporządkowaniu:

\[L_{1-2} = idem \cdot \int_1^2 V^{-n} \cdot dV \]

Pamiętając, że $\int x^a = \frac{x^{a+1}}{a+1} + C$ rozwiązujemy całkę otrzymując:

\[L_{1-2} = idem \cdot \left( \frac{V_2 ^{1-n}}{1-n} - \frac{V_1 ^{1-n}}{1-n} \right)\]

Po uporządkowaniu:

\[L_{1-2} = \frac{idem}{1-n} \cdot \left( V_2 ^{1-n} - V_1 ^{1-n} \right)\]

Uwzględniając, że $x^{1-a} = x^1 \cdot x^{-a}$ możemy zapisać:

\[L_{1-2} = \frac{idem}{1-n} \cdot \left( V_2 \cdot V_2^{-n} - V_1 \cdot V_1^{-n} \right)\]

Uporządkowując zapis do postaci...

\[L_{1-2} = \frac{1}{1-n} \cdot \left( V_2 \cdot \frac {idem}{V_2^{n}} - V_1 \cdot \frac{idem}{V_1^{n}} \right)\]

Oraz podstawiając równanie \eqref{eq:number_8} rówanie przyjmuje postać:

\[L_{1-2} = \frac{1}{1-n} \cdot \left( p_2 \cdot V_2 - p_1 \cdot V_1 \right)\]

Co po wyciągnięciu z nawiasu $-1$ daje ostatecznie równanie na pracę przemiany politropowej:

\[L_{1-2} = \frac{1}{n-1} \cdot \left( p_1 \cdot V_1 - p_2 \cdot V_2 \right)\]

Podstawiając odpowiednie wartości z zadania uzyskuje się:

\[L_{1-2} = \frac{1}{n-1} \cdot \left( p_1 \cdot V_1 - p_2 \cdot V_2 \right)\]

Niestety do rozwiązania wciąż brakuje nam informacji o objętości w stanie 2. Na szczęście można ją wyeliminować. W tym celu należy zapisać równanie politropy dla dwóch stanów 1 i 2. 

\[ p_1 \cdot V_1 ^n = p_2 \cdot V_2 ^n \]

Przekształcając:

\[\frac{p_2}{p_1} = \frac{V_1 ^n}{V_2 ^n} \]

Pierwiastkując w stopniu $n$:

\[\left( \frac{p_2}{p_1} \right) ^{\frac{1}{n}} = \frac{V_1}{V_2} \]

\[V_2 = V_1 \cdot \left( \frac{p_2}{p_1} \right) ^{-\frac{1}{n}} \]

Podstawiając do równania na pracę:

\[L_{1-2} = \frac{1}{n-1} \cdot \left( p_1 \cdot V_1 - p_2 \cdot V_1 \cdot \left( \frac{p_2}{p_1} \right) ^{-\frac{1}{n}}  \right)\]

Teraz można wyciągnąć $p_1 \cdot V_1$ przed nawias:

\[L_{1-2} = \frac{1}{n-1} \cdot  p_1 \cdot V_1\cdot \left(1 - V_1 \cdot \frac{p_2}{p_1} \cdot  \left( \frac{p_2}{p_1} \right) ^{-\frac{1}{n}}  \right)\]

Ostatecznie uzyskuje się równanie na pracę przemiany politropowej, do rozwiązania którego nie jest potrzebna znajomość objętości na końcu:

\[L_{1-2} = \frac{1}{n-1} \cdot  p_1 \cdot V_1\cdot \left(1 - \left( \frac{p_2}{p_1} \right) ^{\frac{n-1}{n}}  \right)\]
